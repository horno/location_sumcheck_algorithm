\documentclass{article}
\usepackage[utf8]{inputenc}
\usepackage{graphicx}
\usepackage{tikz}
\usepackage{float}
\usepackage{listings}

%Tot això hauria d'anar en un pkg, però no sé com és fa
\newcommand*{\assignatura}[1]{\gdef\1assignatura{#1}}
\newcommand*{\grup}[1]{\gdef\3grup{#1}}
\newcommand*{\professorat}[1]{\gdef\4professorat{#1}}
\renewcommand{\title}[1]{\gdef\5title{#1}}
\renewcommand{\author}[1]{\gdef\6author{#1}}
\renewcommand{\date}[1]{\gdef\7date{#1}}
\renewcommand{\maketitle}{ %fa el maketitle de nou
    \begin{titlepage}
        \raggedright{UNIVERSITAT DE LLEIDA \\
            Escola Politècnica Superior \\
            Grau en Enginyeria Informàtica\\
            \1assignatura\\}
            \vspace{5cm}
            \centering\huge{\5title \\}
            \vspace{3cm}
            \large{\6author} \\
            \normalsize{\3grup}
            \vfill
            Professorat : \4professorat \\
            Data : \7date
\end{titlepage}}
%Emplenar a partir d'aquí per a fer el títol : no se com es fa el package
%S'han de renombrar totes, inclús date, si un camp es deixa en blanc no apareix

\tikzset{
	%Style of nodes. Si poses aquí un estil es pot reutilitzar més facilment
	pag/.style = {circle, draw=black,
                           minimum width=0.75cm, font=\ttfamily,
                           text centered}
}
\renewcommand{\figurename}{Figura}
\title{Laboratori 5}
\author{Sergi Sales Jové, Sergi Simón Balcells}
\date{Dimarts 19 de Novembre}
\assignatura{Estructures de dades}
\professorat{X. Domingo, J.E. Garrido, JM. Gimeno}
\grup{GM3}

%Comença el document
\begin{document}
\maketitle
\thispagestyle{empty}

\newpage
\pagenumbering{roman}
\tableofcontents
\newpage
\pagenumbering{arabic}

\section{Senderscribe}
\subsection{Disseny recursiu}
\begin{lstlisting}
''' SENDERSCRIBE RECURSIVE '''
FUNCTION raw_input():
    ''' input from file OR stdin '''
    IF len(sys.argv) = 3:
        file_in <- open(sys.argv[1], "r")
        file_in.close()
        RETURN file_in.read()
    ENDIF
    RETURN input()
ENDFUNCTION

FUNCTION encoded_output(encoded_data):
    ''' output to file OR stdout '''
    IF len(sys.argv) = 3:
        file_out <- open(sys.argv[2], "w")
        file_out.write(encoded_data)
        file_out.close()
    ELSE:
        OUTPUT encoded_data
    ENDIF
ENDFUNCTION

FUNCTION encode_pieces(raw_data, checksum, binary_code):
    ''' encode to binary the data AND does the checksum for raw_data '''
    IF raw_data is empty:
        RETURN raw_data, checksum, binary_code
    ENDIF
    character <- raw_data[0]
    checksum += ord(character)
    binary_code += str('{0:02b}'.format(ord(character)%4))
    RETURN encode_pieces(raw_data[1:], checksum, binary_code)
ENDFUNCTION

FUNCTION checksum_code(hex_code, checksum):
    ''' does the checksum for the hexadecimal encoded part '''
    IF hex_code is empty:
        RETURN checksum
    ENDIF
    checksum += ord(hex_code[0])
    RETURN checksum_code(hex_code[1:], checksum)
ENDFUNCTION

MAIN:
    RAW_DATA <- raw_input()
    RAW_DATA <- RAW_DATA.rstrip("\n\r")
    CHECKSUM <- 0
    BINARY_CODE <- "1"
    [], CHECKSUM, BINARY_CODE <- encode_pieces(RAW_DATA, CHECKSUM, BINARY_CODE)
    HEX_CODE <- format(int(BINARY_CODE, 2), 'x').upper()
    CHECKSUM <- checksum_code(HEX_CODE, CHECKSUM)
    ENCODED_DATA <- RAW_DATA + " " + HEX_CODE + " " + 
                    + str(format(CHECKSUM, 'x')).upper()
    encoded_output(ENCODED_DATA)
ENDMAIN
\end{lstlisting}
\newpage
\subsection{Disseny iteratiu}
\begin{lstlisting}
''' SENDERSCRIBE ITERATIVE '''
FUNCTION raw_input():
    ''' input from file OR stdin '''
    IF len(sys.argv) = 3:
        file_in <- open(sys.argv[1], "r")
        file_in.close()
        RETURN file_in.read()
    ENDIF
    RETURN input()
ENDFUNCTION

FUNCTION encoded_output(encoded_data):
    ''' output to file OR stdout '''
    IF len(sys.argv) = 3:
        file_out <- open(sys.argv[2], "w")
        file_out.write(encoded_data)
        file_out.close()
    ELSE:
        OUTPUT encoded_data
    ENDIF
ENDFUNCTION

MAIN:
    RAW_DATA <- raw_input()
    RAW_DATA <- RAW_DATA.rstrip("\n\r")
    CHECKSUM <- 0
    BINARY_CODE <- "1"
    for character in RAW_DATA:
        CHECKSUM += ord(character)
        BINARY_CODE += str('{0:02b}'.format(ord(character)%4))
    ENDFOR
    HEX_CODE <- format(int(BINARY_CODE, 2), 'x').upper()
    for character in HEX_CODE:
        CHECKSUM += ord(character)
    ENDFOR
    ENCODED_DATA <- RAW_DATA + " " + HEX_CODE + " " + 
				+ str(format(CHECKSUM, 'x')).upper()
    encoded_output(ENCODED_DATA)
ENDMAIN
\end{lstlisting}
\subsection{Cost teòric}
\subsection{Cost experimental}

\section{Receiverscribe}
\subsection{Disseny recursiu}
\begin{lstlisting}
''' RECEIVERSCRIBE RECURSIVE '''
FUNCTION encoded_input():
    ''' input from file or stdin '''
    IF len(sys.argv) = 3:
        file_in <- open(sys.argv[1], "r")
        encoded_data <- file_in.read()
        file_in.close()
        RETURN encoded_data.rstrip("\n\r")
    ENDIF
    RETURN input()
ENDFUNCTION

FUNCTION decoded_output(result):
    ''' output to file or stdout '''
    IF len(sys.argv) = 3:
        file_out <- open(sys.argv[2], 'w')
        file_out.write(result)
        file_out.close()
    ELSE:
        OUTPUT result
    ENDIF
ENDFUNCTION

FUNCTION checksum_code(hex_code, checksum):
    ''' checksum of the hex_code part '''
    IF hex_code is empty:
        RETURN checksum
    ENDIF
    checksum += ord(hex_code[0])
    RETURN checksum_code(hex_code[1:], checksum)
ENDFUNCTION

FUNCTION scan_data(raw_data, checksum_calculated, binary_code, counter, location):
    ''' converts AND scans the data in order to detect an error AND its location '''
    IF raw_data is empty:
        RETURN checksum_calculated, location
    ENDIF
    character <- raw_data[0]
    checksum_calculated += ord(character)
    IF ord(character)%4 != int(binary_code[2*counter:2*counter+2], 2):
        checksum_calculated -= ord(character)
        location <- counter
    ENDIF
    RETURN scan_data(raw_data[1:], checksum_calculated, binary_code,
                     counter+1, location)
ENDFUNCTION

IF __name__ = "__main__":
    ENCODED_DATA <- encoded_input()
    WALL_1 <- ENCODED_DATA.rfind(' ')
    WALL_2 <- ENCODED_DATA.rfind(' ', 0, WALL_1)
    try:
        CHECKSUM_PASSED <- int(ENCODED_DATA[WALL_1+1:], 16)
        RAW_DATA <- ENCODED_DATA[0:WALL_2]
        HEX_CODE <- ENCODED_DATA[WALL_2+1:WALL_1]
        INT_CODE <- int(HEX_CODE, 16)
        BINARY_CODE <- str(bin(INT_CODE)[2:])[1:]
    except ValueError as error:
        IF len(sys.argv) = 3:
            FILE_O <- open(sys.argv[2], 'w')
            FILE_O.write("KO")
            FILE_O.close()
            sys.exit()
        ELSE:
            OUTPUT "KO"
            sys.exit()
        ENDIF
    COUNTER <- 0
    LOCATION <- -1
    CHECKSUM_CALCULATED <- 0
    CHECKSUM_CALCULATED, LOCATION <- scan_data(RAW_DATA, CHECKSUM_CALCULATED,
                                              BINARY_CODE, COUNTER, LOCATION)
    CHECKSUM_CALCULATED <- checksum_code(HEX_CODE, CHECKSUM_CALCULATED)
    IF LOCATION != -1:
        CORRECTED_CHARACTER <- chr(CHECKSUM_PASSED - CHECKSUM_CALCULATED)
        RESULT <- "KO\n" + str(LOCATION) + " " + CORRECTED_CHARACTER
    ELSEIF CHECKSUM_PASSED != CHECKSUM_CALCULATED:
        RESULT <- "KO"
    ELSE:
        RESULT <- "OK"
    ENDIF
    decoded_output(RESULT)
\end{lstlisting}
\newpage
\subsection{Disseny iteratiu}
\begin{lstlisting}
''' RECEIVERSCRIBE ITERATIVE '''
FUNCTION encoded_input():
    ''' input from file OR stdin '''
    IF len(sys.argv) = 3:
        file_in <- open(sys.argv[1], "r")
        encoded_data <- file_in.read()
        file_in.close()
        RETURN encoded_data.rstrip("\n\r")
    ENDIF
    RETURN input()
ENDFUNCTION

FUNCTION decoded_output(result):
    ''' output to file OR stdout '''
    IF len(sys.argv) = 3:
        file_out <- open(sys.argv[2], 'w')
        file_out.write(result)
        file_out.close()
    ELSE:
        OUTPUT result
    ENDIF
ENDFUNCTION

MAIN:
    ENCODED_DATA <- encoded_input()
    WALL_1 <- ENCODED_DATA.rfind(' ')
    WALL_2 <- ENCODED_DATA.rfind(' ', 0, WALL_1)
    try:
        CHECKSUM_PASSED <- int(ENCODED_DATA[WALL_1+1:], 16)
        RAW_DATA <- ENCODED_DATA[0:WALL_2]
        HEX_CODE <- ENCODED_DATA[WALL_2+1:WALL_1]
        INT_CODE <- int(HEX_CODE, 16)
        BINARY_CODE <- str(bin(INT_CODE)[2:])[1:]
    except ValueError as error:
        IF len(sys.argv) = 3:
            FILE_O <- open(sys.argv[2], 'w')
            FILE_O.write("KO")
            FILE_O.close()
            sys.exit()
        ELSE:
            OUTPUT "KO"
            sys.exit()
        ENDIF
    COUNTER <- 0
    LOCATION <- -1
    CHECKSUM_CALCULATED <- 0
    for character in RAW_DATA:
        CHECKSUM_CALCULATED += ord(character)
        IF ord(character)%4 != int(BINARY_CODE[2*COUNTER:2*COUNTER+2], 2):
            CHECKSUM_CALCULATED -= ord(character)
            LOCATION <- COUNTER
        ENDIF
        COUNTER += 1
    ENDFOR
    for character in HEX_CODE:
        CHECKSUM_CALCULATED += ord(character)
    ENDFOR
    IF LOCATION != -1:
        CORRECTED_CHARACTER <- chr(CHECKSUM_PASSED - CHECKSUM_CALCULATED)
        RESULT <- "KO\n" + str(LOCATION) + " " + CORRECTED_CHARACTER
    ELSEIF CHECKSUM_PASSED != CHECKSUM_CALCULATED:
        RESULT <- "KO"
    ELSE:
        RESULT <- "OK"
    ENDIF
    decoded_output(RESULT)
ENDMAIN
\end{lstlisting}
\subsection{Cost teòric}
\subsection{Cost experimental}

\end{document}
