\documentclass{article}
\usepackage[utf8]{inputenc}
\usepackage{graphicx}
\usepackage{tikz}
\usepackage{float}
\usepackage{listings}

%Tot això hauria d'anar en un pkg, però no sé com és fa
\newcommand*{\assignatura}[1]{\gdef\1assignatura{#1}}
\newcommand*{\grup}[1]{\gdef\3grup{#1}}
\newcommand*{\professorat}[1]{\gdef\4professorat{#1}}
\renewcommand{\title}[1]{\gdef\5title{#1}}
\renewcommand{\author}[1]{\gdef\6author{#1}}
\renewcommand{\date}[1]{\gdef\7date{#1}}
\renewcommand{\maketitle}{ %fa el maketitle de nou
    \begin{titlepage}
        \raggedright{UNIVERSITAT DE LLEIDA \\
            Escola Politècnica Superior \\
            Grau en Enginyeria Informàtica\\
            \1assignatura\\}
            \vspace{5cm}
            \centering\huge{\5title \\}
            \vspace{3cm}
            \large{\6author} \\
            \normalsize{\3grup}
            \vfill
            Professorat : \4professorat \\
            Data : \7date
\end{titlepage}}
%Emplenar a partir d'aquí per a fer el títol : no se com es fa el package
%S'han de renombrar totes, inclús date, si un camp es deixa en blanc no apareix

\tikzset{
	%Style of nodes. Si poses aquí un estil es pot reutilitzar més facilment
	pag/.style = {circle, draw=black,
                           minimum width=0.75cm, font=\ttfamily,
                           text centered}
}
\renewcommand{\figurename}{Figura}
\title{Pràctica 1}
\author{Ian Palacín Aliana}
\date{Dissabte 07 d'Abril}
\assignatura{Algorítmica i complexitat}
\professorat{Jordi Planes Cid, Aitor Corchero Rodríguez}
\grup{GM3}

%Comença el document
\begin{document}
\maketitle
\thispagestyle{empty}

\newpage
\pagenumbering{roman}
\tableofcontents
\newpage
\pagenumbering{arabic}

\section{Senderscribe}
L'algoritme de senderscribe, una vegada agafades les dades de
terminal o mitjançant un fitxer, el que fa és fer un pseudo-CRC,
és a dir, fa la operació mòdul 4 al codi asci de cada caràcter
de les dades. Cada mòdul el passa a binari, i de la concatenació
de mòduls en binari ho converteix a hexadecimal. Una vegada obtingut
l'hexadecimal, fa una suma de validació al text original més la clau
hexadecimal obtinguda, que també la passa a hexadecimal. Una vegada
obtingudes les dos claus, les concatena a les dades separant-ho
amb espais. Això és el que envia a la sortida.  
\subsection{Disseny recursiu}
\begin{lstlisting}
''' SENDERSCRIBE RECURSIVE '''
FUNCTION raw_input():
    ''' input from file OR stdin '''
    IF len(sys.argv) = 3:
        file_in <- open(sys.argv[1], "r")
        file_in.close()
        RETURN file_in.read()
    ENDIF
    RETURN input()
ENDFUNCTION

FUNCTION encoded_output(encoded_data):
    ''' output to file OR stdout '''
    IF len(sys.argv) = 3:
        file_out <- open(sys.argv[2], "w")
        file_out.write(encoded_data)
        file_out.close()
    ELSE:
        OUTPUT encoded_data
    ENDIF
ENDFUNCTION

FUNCTION encode_pieces(raw_data, checksum, binary_code):
    ''' encode to binary the data AND does the checksum for raw_data '''
    IF raw_data is empty:
        RETURN raw_data, checksum, binary_code
    ENDIF
    character <- raw_data[0]
    checksum += ord(character)
    binary_code += str('{0:02b}'.format(ord(character)%4))
    RETURN encode_pieces(raw_data[1:], checksum, binary_code)
ENDFUNCTION

FUNCTION checksum_code(hex_code, checksum):
    ''' does the checksum for the hexadecimal encoded part '''
    IF hex_code is empty:
        RETURN checksum
    ENDIF
    checksum += ord(hex_code[0])
    RETURN checksum_code(hex_code[1:], checksum)
ENDFUNCTION

MAIN:
    RAW_DATA <- raw_input()
    RAW_DATA <- RAW_DATA.rstrip("\n\r")
    CHECKSUM <- 0
    BINARY_CODE <- "1"
    [], CHECKSUM, BINARY_CODE <- encode_pieces(RAW_DATA, CHECKSUM, BINARY_CODE)
    HEX_CODE <- format(int(BINARY_CODE, 2), 'x').upper()
    CHECKSUM <- checksum_code(HEX_CODE, CHECKSUM)
    ENCODED_DATA <- RAW_DATA + " " + HEX_CODE + " " + 
                    + str(format(CHECKSUM, 'x')).upper()
    encoded_output(ENCODED_DATA)
ENDMAIN
\end{lstlisting}
\newpage
\subsection{Disseny iteratiu}
\begin{lstlisting}
''' SENDERSCRIBE ITERATIVE '''
FUNCTION raw_input():
    ''' input from file OR stdin '''
    IF len(sys.argv) = 3:
        file_in <- open(sys.argv[1], "r")
        file_in.close()
        RETURN file_in.read()
    ENDIF
    RETURN input()
ENDFUNCTION

FUNCTION encoded_output(encoded_data):
    ''' output to file OR stdout '''
    IF len(sys.argv) = 3:
        file_out <- open(sys.argv[2], "w")
        file_out.write(encoded_data)
        file_out.close()
    ELSE:
        OUTPUT encoded_data
    ENDIF
ENDFUNCTION

MAIN:
    RAW_DATA <- raw_input()
    RAW_DATA <- RAW_DATA.rstrip("\n\r")
    CHECKSUM <- 0
    BINARY_CODE <- "1"
    for character in RAW_DATA:
        CHECKSUM += ord(character)
        BINARY_CODE += str('{0:02b}'.format(ord(character)%4))
    ENDFOR
    HEX_CODE <- format(int(BINARY_CODE, 2), 'x').upper()
    for character in HEX_CODE:
        CHECKSUM += ord(character)
    ENDFOR
    ENCODED_DATA <- RAW_DATA + " " + HEX_CODE + " " + 
				+ str(format(CHECKSUM, 'x')).upper()
    encoded_output(ENCODED_DATA)
ENDMAIN
\end{lstlisting}
\newpage
\subsection{Cost teòric recursiu}
L'algoritme recursiu de senderscribe té dos funcions recursives i 0 bucles.
Podem dir que \textbf{n} és la llargada de l'string raw\textunderscore data, 
i que \textbf{cx} és el cost de la línia de codi x. Donades aquestes 
condicions tenim que:\\
\textbf{c1*(n-1) + c2*(n-1) + c3*1 + c4*(n-1) + c5*(n-1) + c6*(n-1) + c7*(n-1)}
\begin{lstlisting}
1 def encode_pieces(raw_data, checksum, binary_code):
2     if not raw_data:
3          return raw_data, checksum, binary_code
4     character = raw_data[0]
5     checksum += ord(character)
6     binary_code += str('{0:02b}'.format(ord(character)%4))
7     return encode_pieces(raw_data[1:], checksum, binary_code)
\end{lstlisting}
I pel segon bucle, on \textbf{n} és la llargada de hex\textunderscore code:\\
\textbf{c1*(n-1) + c2*(n-1) + c3*1 + c4*(n-1) + c5*(n-1)}
\begin{lstlisting}
1 def checksum_code(hex_code, checksum):
2     if not hex_code:
3         return checksum
4     checksum += ord(hex_code[0])
5     return checksum_code(hex_code[1:], checksum)
\end{lstlisting}
Dont de que cada funció es crida des del main una única vegada, que no
hi ha cap bucle, i que el cost de les crides \textbf{cx} són inferiors
a O(n), tenim que el cost de la part recursiva de senderscribe pertany
a \textbf{O(n)}
\subsection{Cost teòric iteratiu}
L'algoritme de senderscribe iteratiu consta de 2 bucles, tenim que
pel primer bucle, el cost teòric seria:\\
\textbf{c1*(n+1) + c2*n + c3*n}
\begin{lstlisting}
1 for character in RAW_DATA:
2        CHECKSUM += ord(character)
3        BINARY_CODE += str('{0:02b}'.format(ord(character)%4))
\end{lstlisting}
En segon lloc, pel següent bucle, seria: \\
\textbf{c1*(n+1) + c2*n}
\begin{lstlisting}
1    for character in HEX_CODE:
2        CHECKSUM += ord(character)
\end{lstlisting}
Assumint que les operacions \textbf{cx} (c1,c2,...) tenen un 
cost inferior a O(n), el primer bucle pertany a \textbf{O(n)}, 
al igual que el segon, fent així que el cost de senderscribe 
iteratiu sigui d'\textbf{O(n)}

\subsection{Cost experimental}
Per raons de problemes amb la llibreria matplotlib a última hora,
no he pogut col·locar els gràfics amb els temps dels 
costos experimentals.
\newpage
\section{Receiverscribe}
Receiverscribe torna a convertir les dades obtingudes a les claus 
parlades anteriorment, i tornant a fer el mòdul pot saber si hi ha hagut
un error i on.
\subsection{Disseny recursiu}
\begin{lstlisting}
''' RECEIVERSCRIBE RECURSIVE '''
FUNCTION encoded_input():
    ''' input from file or stdin '''
    IF len(sys.argv) = 3:
        file_in <- open(sys.argv[1], "r")
        encoded_data <- file_in.read()
        file_in.close()
        RETURN encoded_data.rstrip("\n\r")
    ENDIF
    RETURN input()
ENDFUNCTION

FUNCTION decoded_output(result):
    ''' output to file or stdout '''
    IF len(sys.argv) = 3:
        file_out <- open(sys.argv[2], 'w')
        file_out.write(result)
        file_out.close()
    ELSE:
        OUTPUT result
    ENDIF
ENDFUNCTION

FUNCTION checksum_code(hex_code, checksum):
    ''' checksum of the hex_code part '''
    IF hex_code is empty:
        RETURN checksum
    ENDIF
    checksum += ord(hex_code[0])
    RETURN checksum_code(hex_code[1:], checksum)
ENDFUNCTION

FUNCTION scan_data(raw_data, checksum_calculated, binary_code, counter, location):
    ''' converts AND scans the data in order to detect an error AND its location '''
    IF raw_data is empty:
        RETURN checksum_calculated, location
    ENDIF
    character <- raw_data[0]
    checksum_calculated += ord(character)
    IF ord(character)%4 != int(binary_code[2*counter:2*counter+2], 2):
        checksum_calculated -= ord(character)
        location <- counter
    ENDIF
    RETURN scan_data(raw_data[1:], checksum_calculated, binary_code,
                     counter+1, location)
ENDFUNCTION

IF __name__ = "__main__":
    ENCODED_DATA <- encoded_input()
    WALL_1 <- ENCODED_DATA.rfind(' ')
    WALL_2 <- ENCODED_DATA.rfind(' ', 0, WALL_1)
    try:
        CHECKSUM_PASSED <- int(ENCODED_DATA[WALL_1+1:], 16)
        RAW_DATA <- ENCODED_DATA[0:WALL_2]
        HEX_CODE <- ENCODED_DATA[WALL_2+1:WALL_1]
        INT_CODE <- int(HEX_CODE, 16)
        BINARY_CODE <- str(bin(INT_CODE)[2:])[1:]
    except ValueError as error:
        IF len(sys.argv) = 3:
            FILE_O <- open(sys.argv[2], 'w')
            FILE_O.write("KO")
            FILE_O.close()
            sys.exit()
        ELSE:
            OUTPUT "KO"
            sys.exit()
        ENDIF
    COUNTER <- 0
    LOCATION <- -1
    CHECKSUM_CALCULATED <- 0
    CHECKSUM_CALCULATED, LOCATION <- scan_data(RAW_DATA, CHECKSUM_CALCULATED,
                                              BINARY_CODE, COUNTER, LOCATION)
    CHECKSUM_CALCULATED <- checksum_code(HEX_CODE, CHECKSUM_CALCULATED)
    IF LOCATION != -1:
        CORRECTED_CHARACTER <- chr(CHECKSUM_PASSED - CHECKSUM_CALCULATED)
        RESULT <- "KO\n" + str(LOCATION) + " " + CORRECTED_CHARACTER
    ELSEIF CHECKSUM_PASSED != CHECKSUM_CALCULATED:
        RESULT <- "KO"
    ELSE:
        RESULT <- "OK"
    ENDIF
    decoded_output(RESULT)
\end{lstlisting}
\newpage
\subsection{Disseny iteratiu}
\begin{lstlisting}
''' RECEIVERSCRIBE ITERATIVE '''
FUNCTION encoded_input():
    ''' input from file OR stdin '''
    IF len(sys.argv) = 3:
        file_in <- open(sys.argv[1], "r")
        encoded_data <- file_in.read()
        file_in.close()
        RETURN encoded_data.rstrip("\n\r")
    ENDIF
    RETURN input()
ENDFUNCTION

FUNCTION decoded_output(result):
    ''' output to file OR stdout '''
    IF len(sys.argv) = 3:
        file_out <- open(sys.argv[2], 'w')
        file_out.write(result)
        file_out.close()
    ELSE:
        OUTPUT result
    ENDIF
ENDFUNCTION

MAIN:
    ENCODED_DATA <- encoded_input()
    WALL_1 <- ENCODED_DATA.rfind(' ')
    WALL_2 <- ENCODED_DATA.rfind(' ', 0, WALL_1)
    try:
        CHECKSUM_PASSED <- int(ENCODED_DATA[WALL_1+1:], 16)
        RAW_DATA <- ENCODED_DATA[0:WALL_2]
        HEX_CODE <- ENCODED_DATA[WALL_2+1:WALL_1]
        INT_CODE <- int(HEX_CODE, 16)
        BINARY_CODE <- str(bin(INT_CODE)[2:])[1:]
    except ValueError as error:
        IF len(sys.argv) = 3:
            FILE_O <- open(sys.argv[2], 'w')
            FILE_O.write("KO")
            FILE_O.close()
            sys.exit()
        ELSE:
            OUTPUT "KO"
            sys.exit()
        ENDIF
    COUNTER <- 0
    LOCATION <- -1
    CHECKSUM_CALCULATED <- 0
    for character in RAW_DATA:
        CHECKSUM_CALCULATED += ord(character)
        IF ord(character)%4 != int(BINARY_CODE[2*COUNTER:2*COUNTER+2], 2):
            CHECKSUM_CALCULATED -= ord(character)
            LOCATION <- COUNTER
        ENDIF
        COUNTER += 1
    ENDFOR
    for character in HEX_CODE:
        CHECKSUM_CALCULATED += ord(character)
    ENDFOR
    IF LOCATION != -1:
        CORRECTED_CHARACTER <- chr(CHECKSUM_PASSED - CHECKSUM_CALCULATED)
        RESULT <- "KO\n" + str(LOCATION) + " " + CORRECTED_CHARACTER
    ELSEIF CHECKSUM_PASSED != CHECKSUM_CALCULATED:
        RESULT <- "KO"
    ELSE:
        RESULT <- "OK"
    ENDIF
    decoded_output(RESULT)
ENDMAIN
\end{lstlisting}
\newpage
\subsection{Cost teòric recursiu}
En la part recursiva de reciverscribe tenim:\\
\textbf{c1*(n-1) + c2*(n-1) + c3*1 + c4*(n-1) + c5*(n-1)}
\begin{lstlisting}
1 def checksum_code(hex_code, checksum):
2      if not hex_code:
3         return checksum
4     checksum += ord(hex_code[0])
5     return checksum_code(hex_code[1:], checksum)
\end{lstlisting}
I pel segon bucle:\\
\textbf{c1*(n-1) + c2*(n-1) + c3*1 + c4*(n-1) + c5*(n-1) + c6*(n-1) +
c7*1 + c8*1 + c9*(n-1)}
\begin{lstlisting}
1 def scan_data(raw_data, checksum_calculated, binary_code, counter, location):
2     if not raw_data:
3         return checksum_calculated, location
4     character = raw_data[0]
5     checksum_calculated += ord(character)
6     if ord(character)%4 != int(binary_code[2*counter:2*counter+2], 2):
7         checksum_calculated -= ord(character)
8         location = counter
9     return scan_data(raw_data[1:], checksum_calculated, binary_code,
                     counter+1, location)
\end{lstlisting}
De la mateixa manera que en la part recursiva de l'apartat anterior,
com que les funcions només es criden una única vegada des del main,
el cost serà, de nou, \textbf{O(n)}
\subsection{Cost teòric iteratiu}
En la part iterativa del receiverscribe tenim també dos bucles.
Donades les condicions establertes en la part iterativa de 
l'apartat anterior on tenim que:\\
\textbf{c1*(n+1) + c2*n + c3*n + c4*1 + c5*1 * c6*n}
\begin{lstlisting}
1 for character in RAW_DATA:
2         CHECKSUM_CALCULATED += ord(character)
3         if ord(character)%4 != int(BINARY_CODE[2*COUNTER:2*COUNTER+2], 2):
4             CHECKSUM_CALCULATED -= ord(character)
5             LOCATION = COUNTER
6         COUNTER += 1
\end{lstlisting}
I pel segon bucle:\\
\textbf{c1*(n+1) + c2*n}
\begin{lstlisting}
1    for character in HEX_CODE:
2        CHECKSUM_CALCULATED += ord(character)
\end{lstlisting}
De igual forma que en la part de senderscribe, assumint que el cost 
de \textbf{cx} és inferior a O(n), tenim que la part iterativa de 
receiverscribe pertan a \textbf{O(n)}

\end{document}
